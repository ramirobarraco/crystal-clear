\setchapterimage[6cm]{crystal-racket-logo-01}
\setchapterpreamble[u]{\margintoc}
\chapter{Crystal y Racket}
\labch{crystal-racket}

\section{Resumen de cualidades de Crystal}

Algunas cualidades importantes de nombrar  de crystal son: 

Es compilado esto le permite hacer un chequeo de tipos previo al tiempo de ejecucion y ofrece una mayor a velocidad.

Su sintaxis esta basada fuertemente en otro lenguaje llamado Ruby

A pesar de tener tipado estatico no es necesario definir los tipos de las variables, debido a su inferenciador de tipos. Esto lo logra atra vez de union types y type narrowing

\subsection[]{Union type y Type narrowing}

 En crystal las variables o expresiones pueden consistir de multiples tipos. A esto lo llamamos union types, por ejemplo.

\begin{verbatim}

  if 1 + 2 == 3
  x = 1
else
  x = "hello"
end

x : Int32 | String

\end{verbatim}

En este caso x tiene dos posibles tipos Int32 en el caso que x = 1 o String en caso de que x = "hello".
 Esto genera un problema al momento de evaluar ciertas expresiones, tengamos  en consideracion ahora que usamos la variable x de la siguiente forma

 \begin{verbatim}

  x = x + 3

\end{verbatim}

Dependiendo del valor de x dicha expresion no puede ser evaluada, para esto se agrega un concepto llamado type narrowing. Es posible hacer un chequeo previo del tipo de esta forma

\begin{verbatim}

if x.is_a?(Int32)
  x = x + 3
end

\end{verbatim}

Crystal nos asegura que  el valor de x luego del if es un Int32 y lo cual nos permite que la expresion pueda ser evaluada.
Es importante aclarar que los union type son asignados en tiempo de compilacion y que el chequeo de tipos que nos permite hacer type narrowing ocurre en la ejecucion.




\section{¿Que es racket/Redex y porque lo usamos?}

Racket es un lenguaje basado en lisp y descendiente de scheme. Esta diseñado como una plataforma para implementar y desarollar lenguajes de programacion.
Esta siendo desarollado por el grupo PLT fundado por Matthias Felleisen a mediados de los 90.
 Actualmente funcin como un projecto para crear un ambiente de programacion pedagogico.

 Racket incluye distintas herramientas como macros, modulos, llamadas a la cola, cerraduras lexicas, continuaciones delimitadas, contratos, green threads, Os threads.
 Aunque la cualidad que mas diferencia racket de otros lenguajes basados en lisp es la habilidad de extender el lenguaje, es decir la capacidad de construir nuevos lenguajes
 ya sean de uso global o especifico del dominio. Esto permite la creacion de un modulo llamado PLT Redex,
 un lenguaje de uso especifico para especificar y debuggear  semanticas operacionales. Con tan solo escribir la gramatica  y las reglas de recuccion, y redex te permite
 probar interactivamente los terminos y generar test aleatorios para intentar generar  contraejemplos. 

 Durante este trabajo usamos estas cualidades de racket para formalizar y intentar falsificar propiedades como progreso. Tanto de manera automatica como con nuestro propio 
 suite de test basado en el que usan en la especificacion de Crystal.
