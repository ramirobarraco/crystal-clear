\setchapterpreamble[u]{\margintoc}
\chapter{Introducción}
\labch{intro}

\section{Objetivos}

En esta tesis se trabaja con la intención de definir un lenguaje formal llamado Crystal-Clear
 que modele parte de Crystal. En especial buscamos definir algunas funciones
 básicas en cualquier lenguaje de programación y luego concentrarnos en el algoritmo de inferencia
 de tipos. Para especificar un lenguaje de programación primero es
 importante entender que lo que se busca es describir la sintaxis y semántica del lenguaje
, esto se logra a través de la utilización de diferentes herramientas que se verán a lo largo
 de este trabajo.
 
Otro objetivo de esta tesis es testear dicha formalización y convencernos de que
 Crystal-Clear realmente modela la parte del lenguaje que decidimos trabajar.
 Una vez estemos seguros de esto podremos utilizar herramientas como redex-check
 para probar que esté inferidor cumple con la propiedad safety(definida en pierce).

 \section{Motivación}
 
 Crystal es un lenguaje de programación que fue lanzado en 2021, como tal todavía
 tiene un largo camino por recorrer en temas como comunidad y desarrollo. Para ello
 un paso importante es conseguir una formalización completa de su lenguaje.
 En especial el inferidor de tipos que agrega nuevas cualidades no observadas en otros
 lenguajes ya trabajados.

\section{Herramientas}

En el trabajo de esta tesis se utilizaron múltiples herramientas tanto prácticas
 como teóricas. En la parte teórica utilizamos conceptos como gramáticas, (TODO).
 En cuanto a herramientas prácticas empleamos un lenguaje de programación
 llamado Racket basado en Lisp. La utilización de esta herramienta es debido a
 la existencia  de su librería redex, que nos permite definir estos objetos teóricos
 y usar las mismas para probar propiedades y desarrollar tests para convencernos de
 los resultados obtenidos.

\section{Estructura de la tesis}

La tesis estará dividida en 6 capítulos.

En el primero hablaremos de Crystal
 y Racket/Redex, Explicaremos propiedades interesantes de ambos lenguajes y
 se intentará transmitir un entendimiento acerca de las sintaxis de ambos.

 En el segundo se explicarán conceptos teóricos utilizados que servirán
 para entender el trabajo de la tesis.
 
 En el tercero se hablará de la gramática
 y semántica de Crystal-Clear nuestra formalización.
 
 Luego seguiremos
 a la mecanización donde intentaremos comparar nuestro lenguaje a Crystal,
 observaremos resultados obtenidos y hablaremos de su utilidad para la comunidad,
 también comentaremos sobre carencias de la documentación.
 
 En el penúltimo capítulos
 comentaremos trabajos relacionados, ya sea de otras formalizaciones
 de otros lenguajes como de conceptos teóricos similares desarrollados por otros
 investigadores.
 
 Por último daremos a conocer nuestras conclusiones y posibles
 trabajos futuros.
 
