\setchapterstyle{kao}
\setchapterpreamble[u]{\margintoc}
\chapter{Resumen}

 Crystal es un lenguaje de programación orientado a objetos, diseñado y desarrollado por Ary Borenszweig, Juan Wajnerman, Brian Cardiff y más de 300 colaboradores.
  Con una sintaxis inspirada en el lenguaje Ruby, es un lenguaje compilado con verificación estática de tipos, pero sin la necesidad de especificar tipos de variables o los argumentos de los métodos.
  Los tipos se resuelven mediante un algoritmo de inferencia de tipos globales.
  Crystal presentó su versión 1.0.0 este año y sigue en desarrollo con una comunidad sumamente activa, con más de 6000 Shards (librerias de Crystal).

 En este trabajo intentaremos formalizar una pequeña parte del lenguaje, en particular la inferencia de tipos.
  Utilizaremos PLT Redex para modelar Crystal-Clear y la probaremos con una suite de test de nuestra autoría.
  También usaremos herramientas que Redex ofrece para Random-testing y probaremos propiedades del lenguaje como seguridad.

 Formalizar esta semántica puede otorgarle mayor seguridad y confianza a los proyectos desarrollados en Crystal