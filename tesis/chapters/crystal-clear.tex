\chapter{Crystal-Clear}
\labch{intro}

\section{Especificación}
\begin{figure}[H]
\centering
    \begin{tabbing}
          \NT{P} ::= \= \NT{P P P ...}\\
          \> \sep \NT{e}\\
          \> \sep \NT{s}\\\\
          
          \NT{e} ::= \= \NT{v} \sep \NT{var} \sep \NT{isa? t P}\\
          \> \sep \NT{var = P} \sep \NT{t var = P}\\
          \> \sep \NT{unop P} \sep \NT{P binop P}\\\\

          \NT{s} ::= \= \NT{while P P} \\
          \sep \NT{if P then P else P}\\\\

          \NT{var} ::= \= \NT{Name}\\\\

          \NT{v} ::= \= \NT{nil} \sep \NT{bool} \sep \NT{int32} \sep \NT{str}\\\\

          \NT{t} ::= \= \NT{Nil} \sep \NT{Bool} \sep \NT{Int32} \sep \NT{String} \sep \NT{Union} \sep \NT{Unit}\\\\
          
          \NT{st} ::= \= \NT{Nil} \sep \NT{Bool} \sep \NT{Int32} \sep \NT{String}\\\\

          \NT{union} ::= \= \NT{st st st ...}\\\\

          \NT{bool} ::= \= \NT{true} \sep \NT{false}\\\\

          \NT{int32} ::= \= \NT{integer}\\\\

          \NT{str} ::= \= \NT{string}\\\\

          \NT{binop} ::= \= \NT{shortbinop} \sep \NT{strictbinop}\\\\

          \NT{strictbinop} ::= \= \NT{+} \sep \NT{-} \sep \NT{*} \sep \NT{/}\\
          \> \sep \NT{<} \sep \NT{<=} \sep \NT{>} \sep \NT{>=} \sep \NT{==} \\\\

          \NT{shortbinop} ::= \= \NT{and} \sep \NT{or}\\\\

          \NT{unop} ::= \= \NT{-} \sep \NT{not} \\\\

          \NT{r} ::= \= \NT{(ref natural)}\\\\

          \NT{rp} ::= \= \NT{(r v)}\\\\

          \NT{rn} ::= \= \NT{Name} \NT{r}\\\\

          \NT{$\epsilon$} ::= \= \NT{((Name r) ...)} \\\\

          \NT{$\sigma$} ::= \= \NT{((r v) ...)}\\\\
          
          \NT{$\sigma\epsilon-prog$} ::= \=  $\sigma : \epsilon : P$ \\\\

          \NT{reserved-symbol} \= \NT{·}\\\\

          \NT{Name} ::= \= \NT{variable-not-otherwise-mentioned}\\\\
    \end{tabbing}
  \caption{Lenguaje.}
\label{fig:test}
\end{figure}

\section{Seguridad}

En esta sección intentaremos demostrar que Crystal-Clear cumple con la propiedad
seguridad.

\begin{definition}
\labdef{SEGURIDAD}
    SEGURIDAD: para cualquier termino bien tipado del lenguaje no es
    posible llegar a un estado atascado siguiendo las reglas de reducción.
     Esto significa que el programa no es un valor final pero tampoco se 
    pueda aplicar las reglas de reducción definidas.
\end{definition}

Para ello definiremos dos propiedades.

\begin{definition}
\labdef{Progreso}
    PROGRESO: para cualquier termino si esta bien tipado entonces el termino no es atascado.
\end{definition}
\begin{definition}
\labdef{Preservacion}
    PRESERVACIÓN: para cualquier termino si este esta bien tipado y sigue un paso de las reglas
    de reducción entonces el resultado esta bien tipado.
\end{definition}

Comencemos demostrando progreso. Primero definamos que significa el predicado
con nuestros objetos matemáticos.

\begin{theorem}[Progreso]
    supongamos que P es cerrado y \Typetup{$\Gamma$}{P}{t}
    entonces P es un valor o para algún $\sigma$ y $\epsilon$ tal que
    $\Gamma \vdash \sigma, \epsilon$ existe una única configuración $\sigma$' : $\epsilon$' : P'.  

    \sigmaprog{$\sigma$}{$\epsilon$}{P} \fullarrow \sigmaprog{$\sigma'$}{$\epsilon'$}{P'}.

\end{theorem}
\begin{proof}
Esto se puede probar haciendo inducción en el árbol de la prueba de  \Typetup{$\Gamma$}{P}{t}.
 Para T-NIL, T-BOOL, T-INT32 y T-STRING son valores por lo tanto ya quedan demostrados que cumplen.
% Estos son los arboles mas chicos ya que tienen un solo nivel%

T-NAME: En principio es fácil observar que si P es de la forma \Var{Name}
y \Typetup{$\Gamma$}{\Var{Name}}{t}. Entonces $\Var{Name} \in dom(\Gamma)$. Como $\Gamma 
\vdash \sigma, \epsilon$ por lo tanto  $\Var{Name} \in Dom(\epsilon)$ y su referencia $\Var{ref} \in Dom(\sigma)$. Por lo tanto:

\sigmaprog{$\sigma$}{$\epsilon$}{\Var{Name}} \sigmaprogarrow \sigmaprog{$\sigma$}{$\epsilon$}{v} por Dename.

Donde v es el valor relacionado con \Var{Name}. Por otro lado P = \inhole{E}{\Var{Name}} con E = \hole.
 Entonces aplica:

\sigmaprog{$\sigma$}{$\epsilon$}{[\Var{Name}]} \fullarrow \sigmaprog{$\sigma$}{$\epsilon$}{[v]} por $FWD-\sigma$.

por lemma 3.2.3 sabemos que esta es la única forma de descomponer este P y, dado que demostramos que se pueden aplicar reglas de $\sigma-progs$ solo se usan reglas de esta relación.
restaría demostrar que no hay otra regla aplicable de $\sigma-progs$. Esto se cumple ya que solo una regla es igual a este P.

Sea P igual \Var{Name} = P' entonces cuando decimos 
\Typetup{$\Gamma$}{\Var{Name} = P'}{t} pueden pasar 2 cosas:

    1. $\Gamma$ contiene \Var{Name} por lo tanto la regla que se aplica es
    T-Define.

    2. $\Gamma$ no contiene \Var{Name} por lo tanto la regla que se aplica
    es T-Redefine.

Veamos que en ambos casos se cumple que solo hay un único paso de reducción.

Por hipótesis inductiva P' es un valor o se puede aplicar alguna 
reducción sobre P'. Si P' es un valor v y $\Gamma \vdash \sigma, \epsilon$
 donde $\Gamma$ no contiene a \Var{Name} entonces solo podemos aplicar 
 la regla de Naming y nos queda:

\sigmaprog{$\sigma$}{$\epsilon$}{\Var{Name} = v} \fullarrow \sigmaprog{$\sigma'$}{$\epsilon'$}{nil}

donde $\sigma'$ contiene una nueva referencia con valor v y $\epsilon'$
 contiene a Name con esta nueva referencia.
Si \Var{Name} se encontraba dentro de $\Gamma$ entonces $\sigma'$
 tendría el nuevo valor asociado a la referencia
que ya existía en $\epsilon$ mientras que este quedaría igual.

Si P' no es un valor entonces se puede aplicar una regla de reducción tal que.

 \sigmaprog{$\sigma$}{$\epsilon$}{\inhole{\Var{Name} =}{P'}} \fullarrow \sigmaprog{$\sigma''$}{$\epsilon''$}{\inhole{\Var{Name} =}{P''}} ya sea por $FWD-\sigma$ o FWD-pure.

En esta ocasión conseguimos 2 reducciones para el mismo P, pero es fácil darse cuenta que las configuraciones \sigmaprog{$\sigma$}{$\epsilon$}{P} tienen diferentes $\sigma y \epsilon$

T-CONCAT: Si P = $P_{1}$ $P_{2}$ ... $P_{N}$ es fácil ver que si
 \Typetup{$\Gamma$}{$P_{1}$ $P_{2}$ ... $P_{N}$}{t}
entonces \Typetup{$\Gamma$}{$P_{i}$}{$t_{i}$} para todo i entre 1 y n, $t_{n}$ = t.
 Por hipótesis inductiva $P_{1}$ es un valor o existe una regla de reducción tal que.

\sigmaprog{$\sigma$}{$\epsilon$}{$P_{1}$} \fullarrow \sigmaprog{$\sigma'$}{$\epsilon'$}{$P_{1}$'}

Por lo tanto si $P_{1}$ es un valor entonces P quedaría $P_{2}$ ... $P_{N}$ debido a 

\sigmaprog{$\sigma$}{$\epsilon$}{$P_{1}$ $P_{2}$ ... $P_{N}$}  \progarrow \sigmaprog{$\sigma'$}{$\epsilon'$}{$P_{2}$ ... $P_{N}$} debido a more-e

o $P_{1}$ no es un valor entonces P se puede descomponer en P = \inhole{$P_{2}$ ... $P_{N}$}{$P_{1}$}. Entonces :

\sigmaprog{$\sigma$}{$\epsilon$}{\inhole{$P_{2}$ ... $P_{N}$}{$P_{1}$}} \fullarrow \sigmaprog{$\sigma'$}{$\epsilon'$}{$P_1$' $P_{2}$ ... $P_{N}$}

T-ARITHOP Si P = $P_{1}$ \Var{Arithop} $P_{2}$ Entonces por HI solo tenemos dos casos
 
$P_{1}$ es un valor de tipo Int32 por lo que se puede descomponer en P = \inhole{v \Var{Arithop}}{$P_{2}$}. Entonces

\sigmaprog{$\sigma$}{$\epsilon$}{\inhole{v \Var{Arithop}}{$P_{2}$}} \fullarrow \sigmaprog{$\sigma'$}{$\epsilon'$}{v \Var{Arithop} $P_{2}$'}

$P_{1}$ no es un valor por lo tanto P = \inhole{$\Var{Arithop} P_{2}$}{$P_{1}$}. Entonces

\sigmaprog{$\sigma$}{$\epsilon$}{\inhole{$\Var{Arithop} \; P_{2}$}{$P_{1}$}} \fullarrow \sigmaprog{$\sigma'$}{$\epsilon'$}{$P_{1}$' \Var{Arithop} $P_{2}$}


T-RELOP y T-SHORTBINOP son análogos a T-ARITHOP.

T-NOT Si P = \Var{not} $P_{1}$ sabemos que \Typetup{$\Gamma$}{P}{Bool} y por HI
sabemos que $P_{1}$ es un valor booleano o existe una única regla de reducción tal que.

\sigmaprog{$\sigma$}{$\epsilon$}{$P_{1}$} \fullarrow \sigmaprog{$\sigma'$}{$\epsilon'$}{$P_{1}$'}

y P' = \Var{not} $P_{1}$'

T-NEGATIVE es parecido a T-NOT pero \Typetup{$\Gamma$}{$P_{1}$}{Int32}.

T-ISA? si P = \T{isa?} t $P_{1}$ entonces \Typetup{$\Gamma$}{\T{isa?} t $P_{1}$}{Bool}
y por hipótesis inductiva $P_{1}$ es un valor o existe una única regla de reducción.
Si $P_{1}$ es un valor entonces se aplica la regla de isa? devolviendo un BOOL.
Si $P_{1}$ no es un valor se aplica 

\sigmaprog{$\sigma$}{$\epsilon$}{\inhole{\T{isa?} t}{$P_{1}$}} \fullarrow \sigmaprog{$\sigma'$}{$\epsilon'$}{\T{isa?} t $P_{1}$'}

y P' = \T{isa?} t $P_{1}$'.

T-IF si P = \T{if} $P_{1}$ \T{then} $P_{2}$ \T{else} $P_{3}$ entonces \Typetup{$\Gamma$}{\T{if} $P_{1}$ \T{then} $P_{2}$ \T{else} $P_{3}$}{t}.
Por HI tenemos que $P_{1}$ tiene 2 opciones:

\sigmaprog{$\sigma$}{$\epsilon$}{\T{if}[\![ $P_{1}$ ]\!] \T{then} $P_{2}$ \T{else} $P_{3}$} \fullarrow \sigmaprog{$\sigma'$}{$\epsilon'$}{\T{if} [\![$P_{1}$']\!] \T{then} $P_{2}$ \T{else} $P_{3}$}


\sigmaprog{$\sigma$}{$\epsilon$}{\T{if}[\![ \Var{v} ]\!] \T{then} $P_{2}$ \T{else} $P_{3}$} \fullarrow \sigmaprog{$\sigma'$}{$\epsilon'$}{P'}

Donde P' es un $P_{1} o P_{2}$ dependiendo del valor de \Var{v}. 

T-WHILE si P = \T{while} $P_{1} \; P_{2}$  entonces \Typetup{$\Gamma$}{\T{while} $P_{1} \; P_{2}$}{t}.
Por HI tenemos que $P_{1}$ tiene 1 opción:

\sigmaprog{$\sigma$}{$\epsilon$}{\T{while} $[\![ P_{1} ]\!] \; P_{2}$ } \fullarrow \sigmaprog{$\sigma'$}{$\epsilon'$}{\T{if} $P_{1}$ \T{then} ($P_{2}$ (\T{while} $P_{1} \; P_{2}$)) \T{else} nil)}

\end{proof}

\begin{theorem}[Preservacion]
    si

    \Typetup{$\Gamma$}{P}{t}{$\Gamma'$}

    $\Gamma \vdash \sigma, \epsilon$

    \sigmaprog{$\sigma$}{$\epsilon$}{P}  $\rightarrow$ \sigmaprog{$\sigma'$}{$\epsilon'$}{P'} 

    entonces, para algún $\Gamma' \supseteq  \Gamma$

    \Typetup{$\Gamma'$}{P'}{t}{$\Gamma'$}

    $\Gamma' \vdash \sigma', \epsilon'$

\end{theorem}

\begin{proof}
Esto se puede probar haciendo inducción en todos los posibles programas de nuestra
gramática. Para T-NIL, T-BOOL, T-INT32 y T-STRING son valores por lo tanto no hay un P' del cual preocuparnos.

T-NAME: En principio es fácil observar que si P es de la forma \Var{Name} y \Typetup{$\Gamma$}{\Var{Name}}{t} con $\Gamma \vdash \sigma, \epsilon$.
Por Progreso yo se que hay una única regla que aplica a este caso llamada Dename.

\sigmaprog{$\sigma$}{$\epsilon$}{\Var{Name}}  $\rightarrow$ \sigmaprog{$\sigma'$}{$\epsilon'$}{\Var{v}} 

Sabemos que esta regla no altera $\sigma ni \epsilon$.
Por lo tanto puedo tomar $\Gamma' \sigma' \epsilon'$ como $\Gamma \sigma \epsilon$ respectivamente entonces se cumple que $\Gamma' \supseteq  \Gamma$
y que \Typetup{$\Gamma'$}{\Var{v}}{t} ya que Gamma $\Gamma' \vdash \sigma', \epsilon'$

T-DEFINE/T-REDEFINE: Sea la forma de P igual a  \Var{Name} = $P_{1}$. Si $P_{1}$ no es un valor entonces se cumple por HI que podemos aplicar una regla tal que:

\sigmaprog{$\sigma$}{$\epsilon$}{\Var{Name} = $P_{1}$}  $\rightarrow$ \sigmaprog{$\sigma'$}{$\epsilon'$}{\Var{Name} = $P_{1}$'}

Con $\sigma', \epsilon'$ tal que $\Gamma' \vdash \sigma', \epsilon'$ para algún $\Gamma' \supseteq  \Gamma$ y que $P_{1}$' cumple \Typetup{$\Gamma'$}{$P_{1}$'}{t}.
Estos $\Gamma', \sigma', \epsilon'$ Cumplen también en el caso de P' como \Var{Name} = $P_{1}$'.
En el caso de  que $P_{1}$ es un valor entonces hay 2 posibles reglas que se aplican:
DEFINE es el caso en que \Var{Name} no este en $\Gamma$, por lo tanto, lo agrega creando $\Gamma'$ esto cumple con  $\Gamma' \supseteq  \Gamma$.
REDEFINE es el caso en el \Var{Name} este en $\Gamma$, por lo que cambia el valor, pero que solo cambia a $\Gamma$ si el tipo valor de $P_{1}$ es distinto
al tipo que ya tenia definido en $\Gamma$. En los dos casos $\sigma' \epsilon'$ tendrán estos cambios ya sea agregando la variable y su valor o cambiando el valor
de la variable ya asignada.

T-CONCAT: P = $P_{1}$ $P_{2}$ ... $P_{n}$. En el caso de la concatenación la única regla que se puede aplicar solo se fija en $P_{1}$.
Hay dos casos

$P_{1}$ es un valor en cuyo caso: 

\sigmaprog{$\sigma$}{$\epsilon$}{$P_{1}$ $P_{2}$ ... $P_{n}$}  $\rightarrow$ \sigmaprog{$\sigma$}{$\epsilon$}{ $P_{2}$ ... $P_{n}$}

Podemos ver que al no cambiar los $\sigma \epsilon$ no hay necesidad de que $\Gamma$ cambie.

$P_{1}$ no es un valor en cuyo caso:  

\sigmaprog{$\sigma$}{$\epsilon$}{$P_{1}$ $P_{2}$ ... $P_{n}$}  $\rightarrow$ \sigmaprog{$\sigma'$}{$\epsilon'$}{$P_{1}$' $P_{2}$ ... $P_{n}$}

Sabemos por HI que $\sigma' \epsilon'$ cumplen con $\Gamma' \vdash \sigma', \epsilon'$ y este $\Gamma'$ con 

$\Gamma' \supseteq  \Gamma$ y \Typetup{$\Gamma'$}{P'}{t} con P' = $P_{1}$' $P_{2}$ ... $P_{n}$

T-ARITHOP sea el caso de que P = $P_{1} \Var{Arithop} P_{2}$. Análogo a otras pruebas tenemos dos casos.
El caso en el que $P_{1}$ es un valor v entonces se intenta usar reglas para reducir $P_{2}$, a su vez tenemos otros dos casos.
Si $P_{2}$ es un valor, se ejecuta la operación pero $\sigma, \epsilon, \Gamma$ no reciben cambios asi que se puede usar la misma configuración.
Si $P_{2}$ no es un valor entonces :

\sigmaprog{$\sigma$}{$\epsilon$}{$ v \Var{Arithop} P_{2}$}  $\rightarrow$ \sigmaprog{$\sigma'$}{$\epsilon'$}{$v \Var{Arithop} $} %caso de v + v = 5

Por HI sabemos que $\sigma' \epsilon'$ tiene un $\Gamma' tal que \Gamma' \vdash \sigma', \epsilon'$,
\Typetup{$\Gamma'$}{$P_{2}'$}{t} y $\Gamma' \supseteq  \Gamma$ por lo tanto podemos usar este $\Gamma'$ Para P'

El caso en el que $P_{1}$ no es un valor es similar al caso de que $P_{2}$ no es un valor.

T-RELOP y T-SHORTBINOP son análogos a T-ARITHOP. La única diferencia en el caso de SHORTBINOP, es que puede no ser necesario saber si $P_{2}$ es valor o no.

T-NOT  sea P = \Var{Not} $P_{1}$. Parecido a otros casos si $P_{1}$ es un valor $\sigma, \epsilon, \Gamma$ no cambian.
En el caso de que $P_{1}$ no sea un valor entonces por HI tenemos un paso de ejecución único que nos devuelve nuevos $\sigma', \epsilon', \Gamma'$ y utilizamos esos.

T-NEGATIVE es parecido a T-NOT.

T-ISA? Para P = \T{isa?} t $P_{1}$. De nuevo si $P_{1}$ es un valor entonces la configuraciones no cambian y podemos reutilizar el mismo $\Gamma$.
En el caso de que $P_{1}$ no sea un valor, se ejecuta un paso y eso nos entrega nuevos $\sigma', \epsilon', \Gamma'$.

T-IF en el caso que P = \T{if} $P_{1}$ \T{then} $P_{2}$ \T{else} $P_{3}$. Como \Typetup{$\Gamma$}{P}{t}{$\Gamma'$}.
Entonces existen $\Gamma_{1} y \Gamma_{2}$ tal que \Typetup{$\Gamma_{1}$}${P_{2}$}{t}{$\Gamma_{1}'$}.
 Para este caso debemos tener en cuenta la relación de tipado
definida en IF. Esta relación nos permite hacer reducciones de tipos a partir de $P_{1}$, por lemma sabemos que los $\Gamma_{1} y \Gamma_{2}$
están contenidos en $\Gamma$ y por HI.

$\Gamma_{1} \vdash \sigma_{1}, \epsilon_{1}$

$\Gamma_{2} \vdash \sigma_{2}, \epsilon_{2}$

Entonces 

\sigmaprog{$\sigma_{1}$}{$\epsilon{1}$}{$P_{2}$}  $\rightarrow$ \sigmaprog{$\sigma_{1}'$}{$\epsilon_{1}'$}{$P_{2}$}

\sigmaprog{$\sigma_{2}$}{$\epsilon{2}$}{$P_{3}$}  $\rightarrow$ \sigmaprog{$\sigma_{2}'$}{$\epsilon_{2}'$}{$P_{3}$}

T-WHILE


\end{proof}

queremos presentar de un lemma que habla sobre propiedades de términos que emanan de su estructura pero sin asumir si están bien tipado.
Como no asumimos que el termino esta bien tipado no podemos asumir si es o no un redex.
De todos modos si puede suceder que un termino encaje con el lado izquierdo pero  que no sea un redex:
 es decir que no satisfaga las condiciones laterales que se tienen que satisfacer para que se pueda aplicar una regla semántica.
Nosotros queremos hablar solo de si un termino encaja con el lado izquierda o no. independientemente de el cumplimiento de las condiciones laterales.
(por ejemplo caso True + 1)

Para demostrar el lemma de descomposición único deberemos definir alguna forma de hablar sobre los redex que encontramos en el libro de PLT-redex.

Pero en nuestro caso necesitamos hablar de un concepto mas general que redex ya que no sabemos si cumplimos con condiciones laterales.
Por lo que crearemos un concepto de patrones mas amplio

\begin{definition}
\labdef{PERTENECE}
    Diremos que $P \in \Pat{Pat}$ si para cierto programa P y cierto patron \Pat{Pat}. \Pat{Pat} es el lado izquierdo de una de las reglas de la relación progs y P encaja con \Pat{Pat}
    o para algún patron \sigmaprog{$\sigma$}{$\epsilon$}{\Pat{Pat}} del lado izquierdo de una regla de la relación $\sigma-progs$ P encaja con ese patron \Pat{Pat}.
\end{definition}

Esto nos permite definir el siguiente Lemma.

\begin{lemma}[Unico contexto de evaluacion]
    Todo programa P es un valor, o existe un único contexto de evaluación E talque 
    P = \inhole{E}{P'} donde $P' \in \Pat{Pat}$.
\end{lemma}

\begin{proof}
Se puede probar haciendo inducción en la estructura de los programas.
Los casos bases P = int32 | bool | nil | string ya son valores y no hace falta demostrar nada.

En caso de que P = \Var{Name} solo se puede descomponer en P = \inhole{}{\Var{Name}} por definición de E. Por regla DENAME esta en $\sigma-progs$

En el caso de P igual a \Var{Name} = P'.
Por HI P' es un valor o se puede descomponer en P' = \inhole{E}{P''}
donde $P'' \in \Pat{Pat}$.
Si P' es un valor v entonces P = \inhole{}{\Var{Name} = v}, $\Var{Name} = v \in \Pat{Pat}$ para alguna configuración \sigmaprog{$\sigma$}{$\epsilon$}{\Pat{Pat}}.
Por reglas REASSIGN y ASSIGN de $\sigma-progs$

En el caso P igual a $P_{1}$ $P_{2}$ ... $P_{n}$. Tenemos multiples formas de descomponer pero la única que se encuentra en E es
$[\![P_{1}]\!] \; P_{2} ... P_{n}$. Por HI $P_{1}$ es un valor o existe un único contexto de evaluación E tal que cumpla con estas propiedades.
Por lo tanto queda $[\![P_{1}]\!] \; E' P_{2} ... P_{n}$. En caso de que sea un valor v nos queda $[\![v]\!] \; P_{2} ... P_{n}$ y se aplica la regla de progs more-e.

En el caso P igual $P_{1} \; \Var{Binop} \; P_{2}$ tenemos dos posibles descomposiciones.

P = $[\![P_{1}]\!] \; \Var{Binop} \; P_{2}$ 

P = $\Var{v} \; \Var{Binop} \; [\![P_{2}]\!]$ En caso que $P_{1}$ sea \Var{v}.

Notemos que en ambos caso por Hi se cumple que $P_{1}$ es un valor o encaja en el lado izquierdo de una de las reglas progs o $\sigma-progs$.

En el caso de que P sea $P_{1} \; \Var{Shortbinop} \; P_{2}$ es analógico al caso de \Var{Binop}.

Para P = \Var{Not} $P_{1}$ descomponemos solo tenemos una forma de descomponer P = \inhole{\Var{Not}}{$P_{1}$}.

Para el caso de P = - $P_{1}$ es análogo al caso anterior.

Cuando P es \T{isa?} t $P_{1}$ solo se puede descomponer de la forma P = \inhole{\T{isa?} t}{$P_{1}$}.

En el caso de que P sea \T{if} $P_{1}$ \T{then} $P_{2}$ \T{else} $P_{3}$ solo tiene una forma de descomponer y esa es:

P = \T{if} $[\![P_{1}]\!]$ \T{then} $P_{2}$ \T{else} $P_{3}$

En el caso de que P = \T{while} $P_{1} \; P_{2}$ es análogo al caso del \T{if}.




\end{proof}