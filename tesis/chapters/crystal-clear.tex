\chapter{Crystal-Clear}
\labch{intro}

\section{Especificación}

\section{Seguridad}

En esta sección intentaremos demostrar que Crystal-Clear cumple con la propiedad
seguridad.

\begin{definition}
\labdef{SEGURIDAD}
    SEGURIDAD: para cualquier termino bien tipado del lenguaje no es
    posible llegar a un estado atascado siguiendo las reglas de reducción.
     Esto significa que el programa no es un valor final pero tampoco se 
    pueda aplicar las reglas de reducción definidas.
\end{definition}

Para ello definiremos dos propiedades.

\begin{definition}
\labdef{Progreso}
    PROGRESO: para cualquier termino si esta bien tipado entonces el termino no es atascado.
\end{definition}
\begin{definition}
\labdef{Preservacion}
    PRESERVACIÓN: para cualquier termino si este esta bien tipado y sigue un paso de las reglas
    de reducción entonces el resultado esta bien tipado.
\end{definition}

Comencemos demostrando progreso. Primero definamos que significa el predicado
con nuestros objetos matemáticos.

\begin{theorem}[Progreso]
    supongamos que P es cerrado y \Typetup{$\Gamma$}{P}{t}
    entonces P es un valor o para algún $\sigma$ y $\epsilon$ tal que
    $\Gamma \vdash \sigma, \epsilon$ existe una única configuración $\sigma$' : $\epsilon$' : P'.  

    \sigmaprog{$\sigma$}{$\epsilon$}{P} \fullarrow \sigmaprog{$\sigma'$}{$\epsilon'$}{P'}.

\end{theorem}
\begin{proof}
Esto se puede probar haciendo inducción en el árbol de la prueba de  \Typetup{$\Gamma$}{P}{t}.
 Para T-NIL, T-BOOL, T-INT32 y T-STRING son valores por lo tanto ya quedan demostrados que cumplen.
% Estos son los arboles mas chicos ya que tienen un solo nivel%

T-NAME: En principio es fácil observar que si P es de la forma \Var{Name}
y \Typetup{$\Gamma$}{\Var{Name}}{t}. Entonces $\Var{Name} \in dom(\Gamma)$. Como $\Gamma 
\vdash \sigma, \epsilon$ por lo tanto  $\Var{Name} \in Dom(\epsilon)$ y su referencia $\Var{ref} \in Dom(\sigma)$. Por lo tanto:

\sigmaprog{$\sigma$}{$\epsilon$}{\Var{Name}} \sigmaprogarrow \sigmaprog{$\sigma$}{$\epsilon$}{v} por Dename.

Donde v es el valor relacionado con \Var{Name}. Por otro lado P = \inhole{E}{\Var{Name}} con E = \hole.
 Entonces aplica:

\sigmaprog{$\sigma$}{$\epsilon$}{[\Var{Name}]} \fullarrow \sigmaprog{$\sigma$}{$\epsilon$}{[v]} por $FWD-\sigma$.

por lemma 3.2.3 sabemos que esta es la única forma de descomponer este P y, dado que demostramos que se pueden aplicar reglas de $\sigma-progs$ solo se usan reglas de esta relación.
restaría demostrar que no hay otra regla aplicable de $\sigma-progs$. Esto se cumple ya que solo una regla es igual a este P.

Sea P igual \Var{Name} = P' entonces cuando decimos 
\Typetup{$\Gamma$}{\Var{Name} = P'}{t} pueden pasar 2 cosas:

    1. $\Gamma$ contiene \Var{Name} por lo tanto la regla que se aplica es
    T-Define.

    2. $\Gamma$ no contiene \Var{Name} por lo tanto la regla que se aplica
    es T-Redefine.

Veamos que en ambos casos se cumple que solo hay un único paso de reducción.

Por hipótesis inductiva P' es un valor o se puede aplicar alguna 
reducción sobre P'. Si P' es un valor v y $\Gamma \vdash \sigma, \epsilon$
 donde $\Gamma$ no contiene a \Var{Name} entonces solo podemos aplicar 
 la regla de Naming y nos queda:

\sigmaprog{$\sigma$}{$\epsilon$}{\Var{Name} = v} \fullarrow \sigmaprog{$\sigma'$}{$\epsilon'$}{nil}

donde $\sigma'$ contiene una nueva referencia con valor v y $\epsilon'$
 contiene a Name con esta nueva referencia.
Si \Var{Name} se encontraba dentro de $\Gamma$ entonces $\sigma'$
 tendría el nuevo valor asociado a la referencia
que ya existía en $\epsilon$ mientras que este quedaría igual.

Si P' no es un valor entonces se puede aplicar una regla de reducción tal que.

 \sigmaprog{$\sigma$}{$\epsilon$}{\inhole{\Var{Name} =}{P'}} \fullarrow \sigmaprog{$\sigma''$}{$\epsilon''$}{\inhole{\Var{Name} =}{P''}} ya sea por $FWD-\sigma$ o FWD-pure.

En esta ocasión conseguimos 2 reducciones para el mismo P, pero es fácil darse cuenta que las configuraciones \sigmaprog{$\sigma$}{$\epsilon$}{P} tienen diferentes $\sigma y \epsilon$

T-CONCAT: Si P = $P_{1}$ $P_{2}$ ... $P_{N}$ es fácil ver que si
 \Typetup{$\Gamma$}{$P_{1}$ $P_{2}$ ... $P_{N}$}{t}
entonces \Typetup{$\Gamma$}{$P_{i}$}{$t_{i}$} para todo i entre 1 y n, $t_{n}$ = t.
 Por hipótesis inductiva $P_{1}$ es un valor o existe una regla de reducción tal que.

\sigmaprog{$\sigma$}{$\epsilon$}{$P_{1}$} \fullarrow \sigmaprog{$\sigma'$}{$\epsilon'$}{$P_{1}$'}

Por lo tanto si $P_{1}$ es un valor entonces P quedaría $P_{2}$ ... $P_{N}$ debido a 

\sigmaprog{$\sigma$}{$\epsilon$}{$P_{1}$ $P_{2}$ ... $P_{N}$}  \progarrow \sigmaprog{$\sigma'$}{$\epsilon'$}{$P_{2}$ ... $P_{N}$} debido a more-e

o $P_{1}$ no es un valor entonces P se puede descomponer en P = \inhole{$P_{2}$ ... $P_{N}$}{$P_{1}$}. Entonces :

\sigmaprog{$\sigma$}{$\epsilon$}{\inhole{$P_{2}$ ... $P_{N}$}{$P_{1}$}} \fullarrow \sigmaprog{$\sigma'$}{$\epsilon'$}{$P_1$' $P_{2}$ ... $P_{N}$}

T-ARITHOP Si P = $P_{1}$ \Var{Arithop} $P_{2}$ Entonces por HI solo tenemos dos casos
 
$P_{1}$ es un valor de tipo Int32 por lo que se puede descomponer en P = \inhole{v \Var{Arithop}}{$P_{2}$}. Entonces

\sigmaprog{$\sigma$}{$\epsilon$}{\inhole{v \Var{Arithop}}{$P_{2}$}} \fullarrow \sigmaprog{$\sigma'$}{$\epsilon'$}{v \Var{Arithop} $P_{2}$'}

$P_{1}$ no es un valor por lo tanto P = \inhole{$\Var{Arithop} P_{2}$}{$P_{1}$}. Entonces

\sigmaprog{$\sigma$}{$\epsilon$}{\inhole{$\Var{Arithop} \; P_{2}$}{$P_{1}$}} \fullarrow \sigmaprog{$\sigma'$}{$\epsilon'$}{$P_{1}$' \Var{Arithop} $P_{2}$}


T-RELOP y T-SHORTBINOP son análogos a T-ARITHOP.

T-NOT Si P = \Var{not} $P_{1}$ sabemos que \Typetup{$\Gamma$}{P}{Bool} y por HI
sabemos que $P_{1}$ es un valor booleano o existe una única regla de reducción tal que.

\sigmaprog{$\sigma$}{$\epsilon$}{$P_{1}$} \fullarrow \sigmaprog{$\sigma'$}{$\epsilon'$}{$P_{1}$'}

y P' = \Var{not} $P_{1}$'

T-NEGATIVE es parecido a T-NOT pero \Typetup{$\Gamma$}{$P_{1}$}{Int32}.

T-ISA? si P = \T{isa?} t $P_{1}$ entonces \Typetup{$\Gamma$}{\T{isa?} t $P_{1}$}{Bool}
y por hipótesis inductiva $P_{1}$ es un valor o existe una única regla de reducción.
Si $P_{1}$ es un valor entonces se aplica la regla de isa? devolviendo un BOOL.
Si $P_{1}$ no es un valor se aplica 

\sigmaprog{$\sigma$}{$\epsilon$}{\inhole{\T{isa?} t}{$P_{1}$}} \fullarrow \sigmaprog{$\sigma'$}{$\epsilon'$}{\T{isa?} t $P_{1}$'}

y P' = \T{isa?} t $P_{1}$'.

T-IF si P = \T{if} $P_{1}$ \T{then} $P_{2}$ \T{else} $P_{3}$ entonces \Typetup{$\Gamma$}{\T{if} $P_{1}$ \T{then} $P_{2}$ \T{else} $P_{3}$}{t}.
Por HI tenemos que $P_{1}$ tiene 2 opciones:

\sigmaprog{$\sigma$}{$\epsilon$}{\T{if}[\![ $P_{1}$ ]\!] \T{then} $P_{2}$ \T{else} $P_{3}$} \fullarrow \sigmaprog{$\sigma'$}{$\epsilon'$}{\T{if} [\![$P_{1}$']\!] \T{then} $P_{2}$ \T{else} $P_{3}$}


\sigmaprog{$\sigma$}{$\epsilon$}{\T{if}[\![ \Var{v} ]\!] \T{then} $P_{2}$ \T{else} $P_{3}$} \fullarrow \sigmaprog{$\sigma'$}{$\epsilon'$}{P'}

Donde P' es un $P_{1} o P_{2}$ dependiendo del valor de \Var{v}. 

T-WHILE si P = \T{while} $P_{1} \; P_{2}$  entonces \Typetup{$\Gamma$}{\T{while} $P_{1} \; P_{2}$}{t}.
Por HI tenemos que $P_{1}$ tiene 1 opción:

\sigmaprog{$\sigma$}{$\epsilon$}{\T{while} $[\![ P_{1} ]\!] \; P_{2}$ } \fullarrow \sigmaprog{$\sigma'$}{$\epsilon'$}{\T{if} $P_{1}$ \T{then} ($P_{2}$ (\T{while} $P_{1} \; P_{2}$)) \T{else} nil)}

\end{proof}

\begin{theorem}[Preservacion]
    si

    \Typetup{$\Gamma$}{P}{t}

    $\Gamma \vdash \sigma, \epsilon$

    \sigmaprog{$\sigma$}{$\epsilon$}{P}  $\rightarrow$ 
     \sigmaprog{$\sigma'$}{$\epsilon'$}{P'} 

    entonces, para algún $\Gamma' \supseteq  \Gamma$

    \Typetup{$\Gamma'$}{P'}{t}

    $\Gamma' \vdash \sigma', \epsilon'$

\end{theorem}

\begin{proof}
Esto se puede probar haciendo inducción en todos los posibles programas de nuestra
gramática. Para T-NIL, T-BOOL, T-INT32 y T-STRING son valores por lo tanto no hay un P' del cual preocuparnos.

T-NAME: En principio es fácil observar que si P es de la forma \Var{Name} 
\end{proof}

\begin{definition}
\labdef{PERTENECE}
    Diremos que $P \in \Pat{Pat}$ si para cierto programa P y cierto patron \Pat{Pat}. \Pat{Pat} es el lado izquierdo de una de las reglas de la relación progs y P encaja con \Pat{Pat}
    o para algún patron \sigmaprog{$\sigma$}{$\epsilon$}{\Pat{Pat}} del lado izquierdo de una regla de la relación $\sigma-progs$ P encaja con ese patron \Pat{Pat}.
\end{definition}

\begin{lemma}[Unico contexto de evaluacion]
    Todo programa P es un valor, o existe un único contexto de evaluación E talque 
    P = \inhole{E}{P'} donde $P' \in \Pat{Pat}$.
\end{lemma}

\begin{proof}
Se puede probar haciendo inducción en la estructura de los programas.
Los casos bases P = int32 | bool | nil | string ya son valores y no hace falta demostrar nada.

En caso de que P = \Var{Name} solo se puede descomponer en P = \inhole{}{\Var{Name}} por definición de E. Por regla DENAME esta en $\sigma-progs$

En el caso de P igual a \Var{Name} = P'.
Por HI P' es un valor o se puede descomponer en P' = \inhole{E}{P''}
donde $P'' \in \Pat{Pat}$.
Si P' es un valor v entonces P = \inhole{}{\Var{Name} = v}, $\Var{Name} = v \in \Pat{Pat}$ para alguna configuración \sigmaprog{$\sigma$}{$\epsilon$}{\Pat{Pat}}.
Por reglas REASSIGN y ASSIGN de $\sigma-progs$

En el caso P igual a $P_{1}$ $P_{2}$ ... $P_{n}$. Tenemos multiples formas de descomponer pero la única que se encuentra en E es
$[\![P_{1}]\!] \; P_{2} ... P_{n}$. Por HI P' es un valor o existe un único contexto de evaluación E' tal que cumpla con estas propiedades.
Por lo tanto queda $[\![P_{1}]\!] \; E' P_{2} ... P_{n}$. En caso de que sea un valor v nos queda $[\![v]\!] \; P_{2} ... P_{n}$.

En el caso P igual $P_{1} \; \Var{Binop} \; P_{2}$ tenemos dos posibles descomposiciones.

P = $[\![P_{1}]\!] \; \Var{Binop} \; P_{2}$ 

P = $\Var{v} \; \Var{Binop} \; [\![P_{2}]\!]$ En caso que $P_{1}$ sea \Var{v}.

Notemos que en ambos caso por Hi se cumple que $P_{1}$ es un valor o encaja en el lado izquierdo de una de las reglas progs o $\sigma-progs$.

En el caso de que P sea $P_{1} \; \Var{Shortbinop} \; P_{2}$ es analógico al caso de \Var{Binop}.

Para P = \Var{Not} $P_{1}$ descomponemos solo tenemos una forma de descomponer P = \inhole{\Var{Not}}{$P_{1}$}.

Para el caso de P = - $P_{1}$ es análogo al caso anterior.

Cuando P es \T{isa?} t $P_{1}$ solo se puede descomponer de la forma P = \inhole{\T{isa?} t}{$P_{1}$}.

En el caso de que P sea \T{if} $P_{1}$ \T{then} $P_{2}$ \T{else} $P_{3}$ solo tiene una forma de descomponer y esa es:

P = \T{if} $[\![P_{1}]\!]$ \T{then} $P_{2}$ \T{else} $P_{3}$

En el caso de que P = \T{while} $P_{1} \; P_{2}$ es análogo al caso del \T{if}.




\end{proof}