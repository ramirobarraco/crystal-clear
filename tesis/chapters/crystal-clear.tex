\setchapterpreamble[u]{\margintoc}
\chapter{Crystal-Clear}
\labch{intro}

\section{Especificación}

\section{Seguridad}

En esta seccion intentaremos demostrar que Crystal-Clear cumple con la propiedad
seguridad.

\begin{definition}
\labdef{SEGURIDAD}
    SEGURIDAD: para cualquier termino bien tipado del lenguaje no es
    posible llegar a un estado atascado siguiendo las reglas de reduccion.
     Esto significa que el programa no es un valor final pero tampoco se 
    pueda aplicar las reglas de reduccion definidas.
\end{definition}

Para ello definiremos dos propiedades.

\begin{definition}
\labdef{Progreso}
    PROGRESO: para cualquier termino si esta bien tipado entonces el termino no es atascado.
\end{definition}
\begin{definition}
\labdef{Preservacion}
    PRESERVACION: para cualquier termino si este esta bien tipado y sigue un paso de las reglas
    de reduccion entonces el resultado esta bien tipado.
\end{definition}

Comenzemos demostrando progreso. Primero definamos que significa el predicado
con nuestros objetos matematicos.

\begin{theorem}[Progreso]
    supongamos que P es cerrado y \Typetup{$\Gamma$}{P}{t}
    entonces P es un valor o para algun $\sigma$ y $\epsilon$ tal que
    $\Gamma \vdash \sigma, \epsilon$ existe una unica configuracion $\sigma$' : $\epsilon$' : P'.  

    \sigmaprog{$\sigma$}{$\epsilon$}{P} \fullarrow \sigmaprog{$\sigma'$}{$\epsilon'$}{P'}.

\end{theorem}
\begin{proof}
Esto se puede probar haciendo induccion en todos los posibles programas de nuestra
gramatica. Para T-NIL, T-BOOL, T-INT32 y T-STRING son valores por lo tanto ya quedan demostrados que cumplen.

T-NAME: En principio es facil observar que si P es de la forma \Var{Name}
y \Typetup{$\Gamma$}{\Var{Name}}{t}. Entonces \Var{Name} tiene que estar en $\Gamma$. Como $\Gamma 
\vdash \sigma, \epsilon$ por lo tanto \Var{Name} esta en $\epsilon$ y su referencia esta en $\sigma$.
 Por otro lado P = E [\Var{Name}] tiene E vacio, por lemma solo esta descomposicion existe y solo se puede aplicar la regla Dename.

\sigmaprog{$\sigma$}{$\epsilon$}{\Var{Name}} \sigmaprogarrow \sigmaprog{$\sigma$}{$\epsilon$}{v}.

Donde v es el valor relacionado con \Var{Name}. 

Sea P igual \Var{Name} = P' entonces cuando decimos 
\Typetup{$\Gamma$}{\Var{Name} = P'}{t} pueden pasar 2 cosas:

    1. $\Gamma$ contiene \Var{Name} por lo tanto la regla que se aplica es
    T-Define.

    2. $\Gamma$ no contiene \Var{Name} por lo tanto la regla que se aplica
    es T-Redefine.

Veamos que en ambos casos se cumple que solo hay un unico paso de reduccion.

Por hipotesis inductiva P' es un valor o se puede aplicar alguna 
reduccion sobre P'. Si P' es un valor v y $\Gamma \vdash \sigma, \epsilon$
 donde $\Gamma$ no contiene a \Var{Name} entonces solo podemos aplicar 
 la regla de Naming y nos queda:

\sigmaprog{$\sigma$}{$\epsilon$}{\Var{Name} = v} \fullarrow \sigmaprog{$\sigma'$}{$\epsilon'$}{nil}

donde $\sigma'$ contiene una nueva referencia con valor v y $\epsilon'$
 contiene a Name con esta nueva referencia.
Si \Var{Name} se encontraba dentro de $\Gamma$ entonces $\sigma'$
 tendria el nuevo valor asociado a la referencia
que ya existia en $\epsilon$ mientras que este quedaria igual.

Si P' no es un valor entonces se puede aplicar una regla de reduccion tal que
 \sigmaprog{$\sigma$}{$\epsilon$}{\Var{Name} = P'} \fullarrow \sigmaprog{$\sigma''$}{$\epsilon''$}{\Var{Name} = P''}.

T-CONCAT: Si P = $P_{1}$ $P_{2}$ ... $P_{N}$ es facil ver que si
 \Typetup{$\Gamma$}{$P_{1}$ $P_{2}$ ... $P_{N}$}{t}
entonces \Typetup{$\Gamma$}{$P_{i}$}{$t_{i}$} para todo i entre 1 y n, $t_{n}$ = t
 Por hipotesis inductiva $P_{1}$ es un valor o existe
una regla de reduccion tal que.

\sigmaprog{$\sigma$}{$\epsilon$}{$P_{1}$} \fullarrow \sigmaprog{$\sigma'$}{$\epsilon'$}{$P_{1}$'}

Por lo tanto si $P_{1}$ es un valor entonces P quedaria $P_{2}$ ... $P_{N}$
o $P_{1}$ no es un valor y P = $P_{1}$' $P_{2}$ ... $P_{N}$

T-ARITHOP Si P = $P_{1}$ \Var{Arithop} $P_{2}$ Entonces por HI solo tenemos dos casos
 
$P_{1}$ es un valor de tipo Int32 por lo que se busca hole $P_{2}$

$P_{1}$ no es un valor por lo tanto \sigmaprog{$\sigma$}{$\epsilon$}{$P_{1}$} \fullarrow \sigmaprog{$\sigma'$}{$\epsilon'$}{$P_{1}$'}
y \sigmaprog{$\sigma$}{$\epsilon$}{$P_{1}$ \Var{Arithop} $P_{2}$} \fullarrow \sigmaprog{$\sigma'$}{$\epsilon'$}{$P_{1}$' \Var{Arithop} $P_{2}$}

T-RELOP y T-SHORTBINOP son analogos a T-ARITHOP.

T-NOT Si P = \Var{not} $P_{1}$ sabemos que \Typetup{$\Gamma$}{P}{Bool} y por HI
sabemos que $P_{1}$ es un valor booleano o existe una unica regla de reduccion tal que.

\sigmaprog{$\sigma$}{$\epsilon$}{$P_{1}$} \fullarrow \sigmaprog{$\sigma'$}{$\epsilon'$}{$P_{1}$'}

y P' = \Var{not} $P_{1}$'

T-NEGATIVE es parecido a T-NOT pero \Typetup{$\Gamma$}{$P_{1}$}{Int32}

T-ISA? si P = isa? \Var{t} $P_{1}$ entonces \Typetup{$\Gamma$}{isa? t $P_{1}$}{Bool}
y por hipotesis inductiva $P_{1}$ es un valor o existe una unica regla de reduccion.
Si P es un valor entonces se aplica la regla de isa? devolviendo un BOOL.
Si P no es un valor se aplica 

\sigmaprog{$\sigma$}{$\epsilon$}{$P_{1}$} \fullarrow \sigmaprog{$\sigma'$}{$\epsilon'$}{$P_{1}$'}

y P' = \Var{isa?} \Var{t} $P_{1}$'.

T-IF si P = E [if $P_{1}$ then $P_{2}$ else $P_{3}$] entonces \Typetup{$\Gamma$}{if $P_{1}$ then $P_{2}$ else $P_{3}$}{t}

\end{proof}

\begin{theorem}[Preservacion]
    si

    \Typetup{$\Gamma$}{P}{t}

    $\Gamma \vdash \sigma$

    \sigmaprog{$\sigma$}{$\epsilon$}{P}  $\rightarrow$ 
     \sigmaprog{$\sigma'$}{$\epsilon'$}{P'} 

    entonces, para algun $\Gamma' \supseteq  \Gamma$

    \Typetup{$\Gamma'$}{P'}{t}

    $\Gamma' \vdash \sigma'$

\end{theorem}

\begin{proof}

\end{proof}

\begin{lemma}[Unico contexto de evaluacion]
    Todo progama P es un valor, o existe un unico contexto de evaluacion E talque 
    P = E [P'].
\end{lemma}

\begin{proof}

\end{proof}