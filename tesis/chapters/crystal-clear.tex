\setchapterpreamble[u]{\margintoc}
\chapter{Crystal-Clear}
\labch{intro}

\section{Especificación}

\section{Seguridad}

En esta seccion intentaremos demostrar que Crystal-Clear cumple con la propiedad
seguridad.

\begin{definition}
\labdef{SEGURIDAD}
    SEGURIDAD: para cualquier termino bien tipado del lenguaje no es
    posible llegar a un estado atascado siguiendo las reglas de reduccion.
     Esto significa que el programa no es un valor final pero tampoco se 
    pueda aplicar las reglas de reduccion definidas.
\end{definition}

Para ello definiremos dos propiedades.

\begin{definition}
\labdef{Progreso}
    PROGRESO: para cualquier termino si esta bien tipado entonces el termino no es atascado.
\end{definition}
\begin{definition}
\labdef{Preservacion}
    PRESERVACION: para cualquier termino si este esta bien tipado y sigue un paso de las reglas
    de reduccion entonces el resultado esta bien tipado.
\end{definition}

Comenzemos demostrando progreso. Primero definamos que significa el predicado
con nuestros objetos matematicos.

\begin{theorem}[Progreso]
    supongamos que P es cerrado y bien  tipado(eso es que
    \Typetup{$\Gamma$}{P}{t} para algun t y $\Gamma$)
    entonces P es un valor o para algun $\sigma$ y $\epsilon$ tal que
    $\Gamma \vdash \sigma$ existen P' , $\sigma$' y $\epsilon$' 

    \sigmaprog{$\sigma$}{$\epsilon$}{P} $\rightarrow$ \sigmaprog{$\sigma'$}{$\epsilon'$}{P'}.

\end{theorem}
\begin{proof}
Esto se puede probar haciendo induccion en todos los posibles programas de nuestra
gramatica. Para T-NIL, T-BOOL, T-INT32 y T-STRING la prueba es trivial ya que son valores.
T-NAME En principio es facil observar que si P es de la forma \Var{Name}
y al estar bien tipado entonces \Var{Name} tiene que estar en $\Gamma$. Por lo que lo 
que \Var{Name} tambien estaria en $\sigma, \epsilon$ y el paso que se haria nos devolveria
$\sigma[\epsilon[\Var{Name}]]$
\end{proof}

\begin{theorem}[Preservacion]
    si

    \Typetup{$\Gamma$}{P}{t}

    $\Gamma \vdash \sigma$

    \sigmaprog{$\sigma$}{$\epsilon$}{P}  $\rightarrow$ 
     \sigmaprog{$\sigma'$}{$\epsilon'$}{P'} 

    entonces, para algun $\Gamma' \supseteq  \Gamma$

    \Typetup{$\Gamma'$}{P'}{t}

    $\Gamma' \vdash \sigma'$

\end{theorem}

\begin{proof}

\end{proof}

